\documentclass[
]{scrartcl}

%Deutsche Sprachunterstützung
\usepackage[utf8]{inputenc}
\usepackage[ngerman]{babel}
\usepackage{marvosym}
\DeclareUnicodeCharacter{20AC}{\EUR}

%Für das Einbinden von Bildern
%\usepackage{graphicx}

%Tabellen
\usepackage{array}

%Tabellen automatisch schoener
\usepackage{booktabs}
\usepackage{csquotes}

%Formatierung
\usepackage[T1]{fontenc}
\usepackage{lmodern}
\usepackage{microtype}
\usepackage{circuitikz}
%\usepackage[german=guillemets]{csquotes}
%Formatierungsanweisungen
\newcommand{\wichtig}[1]{\underline{\large{#1}}}
\newcommand{\aref}[1]{(s.Abb. \ref{#1})}
\newcommand{\R}{\mathbb{R}}
\newcommand{\K}{\mathbb{K}}
\newcommand{\C}{\mathbb{C}}


\begin{document}
\title{5. Besprechungrotokoll PPG8}
\maketitle

\paragraph{Anwesende:}
Udo Beier, Leon Brückner, Valentin Olpp, Marco Zech, Sebastian Ziegler, Domenico Tiziani


\section{Status des Projekts}

\subsection{Erstes Versuch}
Das Protokoll wurde fertiggestellt und dem Tutor bereitgestellt. Nach Rücksprache mit dem Tutor wird Valentin Olpp in den kommenden Tagen angesprochene Fehler oder Bemängelungen ausbessern. Bis zum kommenden Montag wird das Protokoll an die Praktikumsleitung weitergegeben. 

\subsection{Zweiter Versuch}
Die Theorie zu dem Interferometer an sich und den Piezoelementen wurde weitgehend erarbeitet. In der letzten Woche wurde das Michelson-Interferometer aufgebaut und in Betrieb genommen. Dabei wird ein 0,8 mW-Laser verwendet, zur Intensitätsmessung wurde bisher eine Photodiode verwendet. Es soll heute getestet werden, ob sich das \enquote{Power-Meter} besser eignet. Bisher konnte allerdings noch keine Interferenz gemessen werden, vielmehr traten viele Fehlerquellen zu Tage, etwa Erschütterungen des optischen Tisches. 

\section{Weiteres Vorgehen}
Zunächst soll nun der Laserstrahl aufgeweitet werden, sodass man Interferenzringe beobachten kann. (Bemerkung von 15.30 Uhr: Gerade wurde Interferenzringe erzielt). Diese Ringe sollen dann mit Hilfe entweder der Photodiode oder dem \enquote{Power-Meter} gemessen werden. In einem zweiten Schritt soll dann die Längenausdehnung eines Piezoelements oder die thermische Ausdehnung eines Stoffes gemessen werden. Weiterhin soll in der kommenden Woche die Theorie zum Interferometer sowie zu Piezoelementen/Längenausdehnung für das Protokoll ausformuliert werden. Dies werden Marco Zech und Sebastian Ziegler übernehmen. Leon Brückner und Udo Beier werden in den folgenden Tagen versuchen, den Aufbau soweit zu verbessern, dass eine Längenänderungsmessung theoretisch möglich ist. 
Anschließend an dieses Projekt ist ein zweiwöchiger Versuch aus der Mechanik oder Thermodynamik geplant, über den bis zum nächsten Dienstag entschieden werden soll. 

\end{document}