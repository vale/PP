\documentclass[
]{scrartcl}

%Deutsche Sprachunterstützung
\usepackage[utf8]{inputenc}
\usepackage[ngerman]{babel}
\usepackage{marvosym}
\DeclareUnicodeCharacter{20AC}{\EUR}

%Für das Einbinden von Bildern
%\usepackage{graphicx}

%Tabellen
\usepackage{array}

%Tabellen automatisch schoener
\usepackage{booktabs}

%Formatierung
\usepackage[T1]{fontenc}
\usepackage{lmodern}
\usepackage{microtype}
\usepackage{circuitikz}
%\usepackage[german=guillemets]{csquotes}
%Formatierungsanweisungen
\newcommand{\wichtig}[1]{\underline{\large{#1}}}
\newcommand{\aref}[1]{(s.Abb. \ref{#1})}
\newcommand{\R}{\mathbb{R}}
\newcommand{\K}{\mathbb{K}}
\newcommand{\C}{\mathbb{C}}


\begin{document}
\title{8. Besprechungsprotokoll PPG8}
\date{03.12.2013}
\maketitle

\paragraph{Anwesende:}
Udo Beier, Leon Br"uckner, Valentin Olpp, Marco Zech, Sebastian Ziegler, Domenico Tiziani


\section{Status des Projekts}

\subsection{Momentaner Status 2. Versuch}
Messung des Interferometers ist abgeschlossen. Das Protokoll wurde angefangen. Das Protokoll wird am 09.12 an den Tutor geschickt.

\subsection{Weiteres Vorgehen 3. Versuch}
Einigung auf den Bau eines Doppelpendels. Doppel Pendel soll aus Acryll im FabLab der FAU hergestellt werden. Messung der Bewegung soll mit einer Kamera durchgeführt werden. Die Massen sollen mit LEDs markiert werden um sie besser mit der Kamera erkennen zu können. Eine stabile Aufhängung muss während der Woche gefunden und gebaut werden.

\subsection{Arbeitsaufteilung}
Udo Beier: Schreiben der Auswertung des Protokolls zu Versuch 2
\\
Leon Br"uckner: Schreiben der Theorie des Protokolls zu Versuch 2
\\
Valentin Olpp: Letzte Korrekturen am Protokoll zu Versuch 1 umsetzten, herstellen der Pendelbauteile und LED Schaltungen zu Versuch 3
\\
Marco Zech: Besorgung der benötigten Metallteile und Vorbereitung der theoretischen Betrachtung von Versuch 3
\\
Sebastian Ziegler: Beschreibung der Versuchsdurchführung von Versuch 2 für das Protokoll


\end{document}