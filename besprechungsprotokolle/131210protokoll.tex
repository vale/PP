\documentclass[
]{scrartcl}

%Deutsche Sprachunterstützung
\usepackage[utf8]{inputenc}
\usepackage[ngerman]{babel}
\usepackage{marvosym}
\DeclareUnicodeCharacter{20AC}{\EUR}

%Für das Einbinden von Bildern
%\usepackage{graphicx}

%Tabellen
\usepackage{array}

%Tabellen automatisch schoener
\usepackage{booktabs}

%Caption
\usepackage{caption}
\usepackage{subcaption}

%Formeln
\usepackage{mathtools}
\usepackage{amsmath}
\usepackage{amssymb}
\usepackage{amstext}
\usepackage{dsfont}

%\usepackage{mnsymbol}

%Vectorpfeile schöner
\usepackage{esvect}

%Formatierung
\usepackage[T1]{fontenc}
\usepackage{lmodern}
\usepackage{microtype}
%\usepackage[german=guillemets]{csquotes}

%Formatierungsanweisungen
\newcommand{\wichtig}[1]{\underline{\large{#1}}}
\newcommand{\aref}[1]{(s.Abb. \ref{#1})}
\newcommand{\R}{\mathbb{R}}
\newcommand{\K}{\mathbb{K}}
\newcommand{\C}{\mathbb{C}}
\begin{document}

\title{Besprechungs-Protokoll Gruppentreffen PPG8-9}
\date{10. Dezember 2013}
\maketitle

\paragraph*{Anwesende}
Udo Beier,
Leon Brückner,
Valentin Olpp,
Sebastian Ziegler,
Domenico Tiziani


\section{Status des Projekts}

\subsection{Erster Versuch}
Das verbesserte Protokoll wurde der Praktikumsleitung übergeben. 

\subsection{Zweiter Versuch}
Das Protokoll wurde am Montag fertiggestellt und dem Tutor zugeschickt. 

\subsection{Dritter Versuch}
In der letzten Woche wurde das Doppelpendel fertiggestellt. Zur Datenaufnahme soll der Versuch in verdunkeltem Raum mit an den Massepunkten befestigten LEDs aufgenommen werden und nachträglich mit einem Tracking-Programm ausgewertet werden. Das Pendel soll zwischen zwei Tischen aufgehängt werden, um die Energieübertragung an die Aufhängung zu minimieren. 

\section{Weiteres Vorgehen}
Heute soll die tatsächliche Messung zum Einen für kleine Winkel, wobei eine theoretische Betrachtung möglich sein sollte, zum anderen für großere Amplituden, wo deutlich chaotisches Verhalten zu beobachten ist, ausgeführt werden. Marco Zech und Udo Beier werden die theoretische Betrachtung zum Doppelpendel unter Verwendung von Lagrange-Formalismus und inklusive chaotischen System ausarbeiten. Valentin Olpp lötet die zweite Diode an, wird die Durchführung verfassen und Sebastian Ziegler wird die Auswertung durchführen. Leon Brückner schreibt das Vorwort und stellt Überlegungen zur Reibung an. 
\end{document}
