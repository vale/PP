\documentclass[
]{scrartcl}

%Deutsche Sprachunterstützung
\usepackage[utf8]{inputenc}
\usepackage[ngerman]{babel}
\usepackage{marvosym}
\DeclareUnicodeCharacter{20AC}{\EUR}

%Für das Einbinden von Bildern
%\usepackage{graphicx}

%Tabellen
\usepackage{array}

%Tabellen automatisch schoener
\usepackage{booktabs}

%Caption
\usepackage{caption}
\usepackage{subcaption}

%Formeln
\usepackage{mathtools}
\usepackage{amsmath}
\usepackage{amssymb}
\usepackage{amstext}
\usepackage{dsfont}

%\usepackage{mnsymbol}

%Vectorpfeile schöner
\usepackage{esvect}

%Formatierung
\usepackage[T1]{fontenc}
\usepackage{lmodern}
\usepackage{microtype}
%\usepackage[german=guillemets]{csquotes}

%Formatierungsanweisungen
\newcommand{\wichtig}[1]{\underline{\large{#1}}}
\newcommand{\aref}[1]{(s.Abb. \ref{#1})}
\newcommand{\R}{\mathbb{R}}
\newcommand{\K}{\mathbb{K}}
\newcommand{\C}{\mathbb{C}}
\begin{document}

\title{Besprechungs-Protokoll Gruppentreffen PPG8-1}
\date{15. Oktober 2013}
\maketitle

\paragraph*{Anwesende}
Leon Brückner,
Udo Beier,
Sebastian Ziegler,
Marco Zech,
Valentin Olpp
\paragraph*{Betreuer}
Domenico Tiziani

\section{Status des Projekts}
\subsection{Webseite}
Webseite wird mit Wordpress erstellte.
\subsection{Erster Versuch}
\paragraph{Theorie} zu Wechselstromrechnung und Hochpass/Tiefpassfrequenzfilter besprochen.
\section{Weiteres Vorgehen}
\subsection{Erster Versuch}
\begin{itemize}
\item Weiteres Besprechen der Theorie.
\item Überprüfung des Lagers auf passende Bauteile
\item Versuchsaufbau planen und berechnen
\item Versuch der Umsetzung, Überprüfung der Eignung der Geräte
\end{itemize}
\end{document}