\documentclass[
]{scrartcl}

%Deutsche Sprachunterstützung
\usepackage[utf8]{inputenc}
\usepackage[ngerman]{babel}
\usepackage{marvosym}
\DeclareUnicodeCharacter{20AC}{\EUR}

%Für das Einbinden von Bildern
%\usepackage{graphicx}

%Tabellen
\usepackage{array}

\usepackage{csquotes}
%Tabellen automatisch schoener
\usepackage{booktabs}

%Formatierung
\usepackage[T1]{fontenc}
\usepackage{lmodern}
\usepackage{microtype}
\usepackage{circuitikz}
%\usepackage[german=guillemets]{csquotes}
%Formatierungsanweisungen
\newcommand{\wichtig}[1]{\underline{\large{#1}}}
\newcommand{\aref}[1]{(s.Abb. \ref{#1})}
\newcommand{\R}{\mathbb{R}}
\newcommand{\K}{\mathbb{K}}
\newcommand{\C}{\mathbb{C}}


\begin{document}
\title{4. Besprechungrotokoll PPG8}
\maketitle

\paragraph{Anwesende:}
Udo Beier, Leon Brückner, Valentin Olpp, Marco Zech, Sebastian Ziegler, Domenico Tiziani


\section{Status des Projekts}

\subsection{Erstes Versuch}
Das Protokoll ist bis auf das Kapitel \enquote{Diskussion der Ergebnisse} fertiggestellt, wird nun im Laufe der Woche korrekturgelesen und Domenico Tiziani zugesendet. 
Der Versuchsaufbau wurde vollständig abgebaut und in die Lager zurückgebracht. 

\subsection{Zweiter Versuch}
Die Theorie zu dem Interferometer an sich und den Piezoelementen wurde weitgehend erarbeitet. 

\section{Weiteres Vorgehen}
In dieser Woche sollen die Vorversuche stattfinden, in deren Rahmen die Funktionsfähigkeit des Aufbaus verifiziert wird. 
Dazu soll einer der Spiegel mit einem Verfahrtisch verschoben werden und diese Verschiebung quantifiziert werden. 
Als Hauptversuch soll die Längenänderung eines Piezoelements beim Anlegen einer Spannung oder die thermische Ausdehnung 
eines bestimmten Materials gemessen werden.
Schwierigkeiten, die dabei auftreten können, sind die hinreichend feste Befestigung des Spiegels am Piezokristall (bzw. sich thermisch ausdehnenden Material). 


\end{document}