\section{Vorwort}
Frequenzfilter bilden die Grundlage für viele technische Anwendungen. Sie sind insbesondere in der Tontechnik essenziell, wenn es darum geht bestimmte Frequenzen hervorzuheben oder abzuschwächen. Dies kann sehr gut am Beispiel einer Bassdrum verdeutlicht werden. Dort verstärkt man Frequenzen bei ca. 80 Hz, sowie hohe Frequenzen von mehr als 1 kHz, um die charakteristischen Klänge einer Bassdrum hervorzuheben, während man mittlere Frequenzen abschwächt, da diese das Klangbild meist verwaschen klingen lassen. Aufgrund der immensen Bedeutung von Frequenzfiltern, sollen im folgenden ein Sperr- und ein Durchlassfilter gebaut und die zugehörigen gemessenen Frequenzkurven mit den Theoriekurven verglichen werden.
