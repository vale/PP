\section{Versuchsdurchführung}
\subsection{Durchlassfilter als Frequenzfilter}
Zur Überprüfung der Funktionsweise eines Frequenzfilters wurde zunächst die Schaltung aus
Abb. \ref{plan:durchlass} wie in Abb. \ref{img:durchlass} realisiert. Es wurde die Spule \enquote{Leybold 56214} mit 500 Windungen und einer Induktivität von 9 mH verwendet, wobei das Augenmerk auf einer möglichst geringen Windungszahl liegt, um den in der Spule angesiedelten Widerstand möglichst gering zu halten. Dieser liegt für die Spule bei angegebenen 2,5 $ \Omega $, was durch Messung bestätigt wurde. Auf
der anderen Seite musste auch eine hinreichend große Induktivität garantiert werden, damit die Resonanzfrequenz, in deren Umgebung die Messungen stattfanden, in einem gut erfassbaren Bereich lagen. Weiter wurde ein Plastikfolienkondensator der Firma \enquote{WIMA} der Kapazität 0,22 $  \mu F $ sowie in einem zweiten Teilversuch ein Kondensator gleicher Bauart, allerdings mit 4,7 $  \mu F $ Kapazität, verwendet. Da beide Teilversuche die angestrebte Resonanzfrequenz gut trafen, wird in der Auswertung nur der erste Teilversuch dargestellt. Ferner wurden
mit Hilfe eines Widerstandskasten verschiedene Widerstände im Bereich von 1 $ \Omega $ bis 20 $ \Omega $ eingesetzt.
\paragraph{}
Zur Erzeugung der Eingangsspannung wurde zunächst ein Frequenzgenerator der Marke Hameg (Programmable 15 MHz Function Generator, HM8131/2) verwendet, zur Vereinfachung und Automatisierung der Messung wurde dieser allerdings durch das Power-Cassy ersetzt \footnote{vgl. \cite[42f.]{cassy2013manual}} .
Die Messung der über dem Widerstand abfallenden Ausgangsspannung wurde bei einem ersten Aufbau mit einem Oszilloskop (Tektronix TDS 2024 100MHz, PPL28/2/001) durchgeführt. Hierbei stellte sich allerdings heraus, dass die Messungen nicht die gewünschte Genauigkeit liefern konnten. (Siehe Abb. \ref{plot:oszi})
\begin{figure}
	\includegraphics[width=.9\textwidth]{images/plot/oszi.png}
\caption{Sprünge in handnotierten Werten der Messung mit Oszilloskop klar erkennbar}
\label{plot:oszi}
\end{figure}
Um eine glattere Kurve zu erreichen wurde bei der optimierten Durchführung des Experiments stattdessen das Sensor-Cassy genutzt; Vorteil hierbei ist zudem klar die direkte Übertragung der Daten in die Cassy-Software.

\begin{figure}[h]
\centering
    \subcaptionbox{Durchlassfilter\label{img:durchlass}}[.49\linewidth]
            {\includegraphics[width=.495\textwidth]{images/durchlassfilter.jpg}}
    \subcaptionbox{Sperrfilter\label{img:sperr}}[.49\linewidth]
            {\includegraphics[width=.495\textwidth]{images/sperrfilter.jpg}}
\caption{Fotos der Versuchsaufbauten}
\end{figure}

\subsection{Sperrfilter als Frequenzfilter}
Für die Abwandlung des Durchlassfilters zu einem Sperrfilter müssen lediglich Kapazität und Induktivität parallel anstatt in Reihe geschaltet werden, wie in Abb. \ref{plan:sperr} dargestellt. Für diesen Versuch wurde der zweite Kondensator mit einer Kapazität von 4,7 $  \mu F $ sowie daneben eine Spule höherer Induktivität (36 mH bei 1000 Windungen; ebenfalls von "Leybold") verwendet. Grund hierfür war eine erwartetet Senkung der Resonanzfrequenz und eine damit verbundene bessere Messung, da die Abtastrate des Sensor Cassy besser für niedrige Frequenzen geeignet ist. Tatsächlich wurde aber kein qualitativer Unterschied zum Durchlassfilter und ersterer Spule 
bei der Genauigkeit der Messung erzielt. Um für dieses Experiment den teilweise recht schwach ausgebildeten Peak deutlicher sichtbar zu machen, also eine geringere Breite zu erreichen, wurde weiter die Widerstandsbox durch einen einfachen Schichtwiderstand, wie etwa in Abb. \ref{img:sperr} zu sehen,
ersetzt. Grund dafür ist, dass die Widerstandsbox insbesondere bei den kleinsten Widerständen einen zu großen tatsächlichen Widerstand liefert.



