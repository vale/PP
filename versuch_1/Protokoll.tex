\documentclass[
]{scrartcl}

%Deutsche Sprachunterstützung
\usepackage[utf8]{inputenc}
\usepackage[ngerman]{babel}
\usepackage{marvosym}
\DeclareUnicodeCharacter{20AC}{\EUR}

%Für das Einbinden von Bildern
\usepackage{graphicx}

%Tabellen
\usepackage{array}

%Tabellen automatisch schoener
\usepackage{booktabs}

%Caption
\usepackage{caption}
\usepackage{subcaption}

%Formeln
\usepackage{mathtools}
\usepackage{amsmath}
\usepackage{amssymb}
\usepackage{amstext}
\usepackage{dsfont}

%\usepackage{mnsymbol}

%Vectorpfeile schöner
\usepackage{esvect}

%Formatierung
\usepackage[T1]{fontenc}
\usepackage{lmodern}
\usepackage{microtype}

%Schaltbilder malen
\usepackage[europeanresistors,cuteinductors,siunitx]{circuitikz}

%\usepackage[german=guillemets]{csquotes}

%Formatierungsanweisungen
\newcommand{\wichtig}[1]{\underline{\large{#1}}}
\newcommand{\aref}[1]{(s.Abb. \ref{#1})}
\newcommand{\R}{\mathbb{R}}
\newcommand{\K}{\mathbb{K}}
\newcommand{\C}{\mathbb{C}}
\begin{document}

\title{Frequenzfilter}
\subtitle{1. Versuch PPG 8}
\date
%\author{PPG8}
\maketitle
\section{Vorwort}
blabla bulbbaslkdfalskdf

\include{theo}
\section{Versuchsdurchführung}
Nach einigen Messversuchen


Der Schaltkreis hat 2,7 Ohm + dem Messwiderstand.





\begin{figure}[t]
\centering
	\subcaptionbox{Durchlassfilter\label{img:durchlass}}[.4\linewidth]
		{\includegraphics[width=.4\textwidth]{images/durchlassfilter}}
	\subcaptionbox{Sperrfilter\label{img:sperr}}[.4\linewidth]
		{\includegraphics[width=.4\textwidth]{images/sperrfilter}}
\caption{Fotos der Versuchsaufbauten}
\end{figure}
\section{Versuchsdurchführung}
Nach einigen Messversuchen


Der Schaltkreis hat 2,7 Ohm + dem Messwiderstand.
\section{Diskussion der Ergebnisse}






\end{document}