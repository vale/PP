\section{Vorwort}
Das Doppelpendel, ein Pendel, an dessen Pendelmasse ein weiteres Pendel befestigt wird, mag zunächst nicht sonderlich interessant erscheinen. Bei genauerer Betrachtung hat dieser simple mechanische Aufbau jedoch einige faszinierende und sogar für aktuelle Forschung relevante Eigenschaften. Das Doppelpendel ist ein einfaches Beispiel für ein chaotisches System, d.h. ein System, dessen zeitliche Entwicklung sich schon bei kleinen Variationen der Anfangsbedingungen stark ändert. Chaotische Systeme haben eine wichtige Bedeutung in der Natur und sind der Hauptfokus eines Teilgebiets der modernen Physik, der Chaosforschung. Mithilfe der Chaosforschung lassen sich z.B. Bevölkerungsentwicklungen, die Bildung von Verkehrsstaus oder das Wetter untersuchen.
In diesem Versuch soll nun ein Doppelpendel gebaut, die Physik dahinter untersucht und die Bedeutung als chaotisches System am praktischen Beispiel nachvollzogen werden.