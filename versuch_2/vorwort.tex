\section{Vorwort}
Ursprünglich wurde das Michelson-Morley-Interferometer von Albert Abraham Michelson (1852 - 1931) und Edward William 
Morley (1838 - 1923) verwendet, um die Existenz eines Äthers, also eines möglicherweise existierenden Trägermediums von elektromagnetischen Wellen, 
zu überprüfen. Gäbe es einen solchen Äther, so müsste sich das Labor aufgrund der Erdrotation ebenfalls durch den Äther 
bewegen und zwar in wechselnder Richtung. Nach dem in der Theorie beschriebenen Aufbau würde dies eine Änderung des zu 
beobachtenden Interferenzmusters bewirken. Tatsächlich konnte unter korrekter Ausführung dieses Versuches allerdings 
niemals eine solche Verschiebung festgestellt werden, sodass die Idee des Äthers schließlich fallen gelassen werden 
musste. Dementsprechend bewegt sich das Licht unabhängig vom Bezugssystem des Beobachters immer mit der 
Lichtgeschwindigkeit c, was auch als Einsteinsches Relaltivitätsprinzip bekannt ist. Der Grund dafür ist schlicht, 
dass es die wesentliche Abänderung des Verständnisses von Raum und Zeit darstellt, welches schließlich zur 
Lorentz-Transformation und der speziellen Relativitätstheorie führte. Somit kommt diesem Experiment eine herausragende Rolle 
in der Entwicklung der modernen Physik zu. Tatsächlich kann dieser Versuchsaufbau auch dazu verwendet werden, 
Längenänderungen im Bereich von Wellenlängen des sichtbaren Lichts zu messen. Speziell wurde dies von uns dazu genutzt, 
die thermische Ausdehnung eines Metallstabs zu untersuchen. 
